
\documentclass{article}  
\usepackage{graphicx}
\usepackage{polski}
\usepackage[utf8]{inputenc}
\usepackage{amsmath}

\begin{document}
\section{Dzień pierwszy!}
                
\par \textbf{Wpływ Jeży na życie po życiu} \newline
\textit{Roman Lwowski} \newline
To make multiplication visually similar to a fraction, a nested array can be used, for example multiplication of numbers written one below the other.\begin{equation}\frac{    \begin{array}[b]{r}      \left( x_1 x_2 \right)\\      \times \left( x'_1 x'_2 \right)    \end{array}  }{    \left( y_1y_2y_3y_4 \right)  }\end{equation}\newline 
\par \textbf{Jeżowe kołysanki są dobre} \newline
\textit{ } \newline
Istnieje kilkadziesiąt tysięcy[1] serwisów wykorzystujących technologię Wiki. Największym z nich jest angielska Wikipedia, lecz nie jest ona „typową” wiki.Większość wiki nie rozdziela dyskusji od informacji – każda strona przypomina wątek listy dyskusyjnej, który może być edytowany, dzielony i łączony jeśli nadmiernie się rozrośnie. Największe Wiki tego typu to Wiki Wiki Web poświęcona programowaniu i Sensei’s Library poświęcona grze go.Encyklopedia jest bardzo częstym zastosowaniem wiki – w czerwcu 2003 r. 8 z 10 największych wiki stanowiły encyklopedie. Dla większości języków funkcję wiki-encyklopedii spełnia Wikipedia. Polska Wikipedia jest jedną z największych wiki (w maju 2013, 9. pod względem wielkości) i niewątpliwie największą polskojęzyczną wiki.Wiki Wikimedia Foundation w języku polskim to: Wikipedia, Wikisłownik, Wikicytaty, Wikinews, Wikiźródła i Wikibooks.\newline 
\par \textbf{Przepis na ciasto dla Jeży} \newline
\textit{Anna Kozłowska} \newline
If you use them throughout the document, usage of xfrac package is recommended. This package provides : command to create slanted fractions. Usage:\newline 
\par \textbf{Gruz jest jedzony przez cygany.} \newline
\textit{Rafałek Sz.} \newline
Continued fractions should be written using cfrac command\begin{equation}  x = a_0 + \cfrac{12356616316}{a_3           + \cfrac{16565}{a_3           + \cfrac{165465465}{a_9 + \cfrac{1131321}{a_7} } } }\end{equation}\newline 
\end{document}